\documentclass[xcolor=svgnames]{beamer}


%********** beamer settings **********
\usetheme{Madrid} %{default} %{Singapore} %{Warsaw} %{Copenhagen} %{Madrid}
\setbeamertemplate{navigation symbols}{} % remove the navigation symbols
%\useoutertheme{infolines} % use outer theme for footer.
\definecolor{mybackground}{RGB}{204,232,207} % note the uppercase RGB
%\setbeamercolor{normal text}{bg=red!12} % set background color
%\setbeamercolor{normal text}{bg=mybackground} % set background color
\definecolor{commit-color}{RGB}{204,255,204}
\definecolor{commit-border-color}{RGB}{160,254,160}
\definecolor{index-color}{RGB}{204,204,255}
\definecolor{index-border-color}{RGB}{178,178,254}
\definecolor{wt-color}{RGB}{255,204,204}
\definecolor{wt-border-color}{RGB}{255,153,153}
\definecolor{head-color}{RGB}{224,224,224}
\definecolor{head-border-color}{RGB}{192,192,192}
\definecolor{branch-color}{RGB}{255,224,192}
\definecolor{branch-border-color}{RGB}{255,192,128}

%********** fonts settings **********
\usepackage{fontspec,xunicode,xltxtra}
\setmainfont{Times New Roman}
\setsansfont{Arial} % beamer默认使用sans font
%\setmonofont{Source Sans Pro}

%********** new fonts commands **********
%\newfontfamily\SourceSansPro{Source Sans Pro}

%********** Chinese fonts input **********
\usepackage{xeCJK}
\setCJKmainfont{宋体}
\setCJKsansfont{微软雅黑}
\setCJKmonofont{宋体}
\XeTeXlinebreaklocale "zh"
\XeTeXlinebreakskip=0pt plus 1pt minus 0.1pt

%********** using packages **********
\usepackage{metalogo} % for XeTeX, XeLaTeX logo
\usepackage{listings} % for listing source code
% listings settings
\lstset{
  language=C,
  numbers=left,
  numberstyle=\scriptsize,
  frame=leftline,
  xleftmargin=1em,
  flexiblecolumns=false,
  basicstyle=\ttfamily\small,
  breaklines=true,
  extendedchars=true,
  escapechar=\%,
  texcl=true,
  showstringspaces=false,
  %keywordstyle=\bfseries,
  keywordstyle=\color{cyan},
  commentstyle=\tiny\color{blue}\slshape,
  tabsize=4
}

%********** colors and drawing pictures **********
%\usepackage{xcolor} % beamer 自动加载xcolor
\usepackage{graphicx}
\usepackage{tikz,ifthen}
\usetikzlibrary{arrows,shapes,fit,calc,positioning,intersections}
\tikzset{
  % 定义一些新的样式,注意每个样式定义后面的逗号
  box/.style={
    rectangle,
    rounded corners=3pt,
    minimum height=1.5em,
    minimum width=3em,
    inner sep=3pt,
    draw=gray,thick,
    fill=lightgray
  },
  format/.style = {draw, thin, fill=blue!20},
  medium/.style = {
    ellipse, draw, thin,
    fill=green!20,
    minimum height=2.5em
  },
  git-commit/.style = {
    rectangle,
    rounded corners=3pt,
    minimum height=1.5em,
    minimum width=3em,
    inner sep=3pt,
    draw=commit-border-color,
    line width=1.5pt,
    fill=commit-color
  },
  git-index/.style = {
    rectangle,
    rounded corners=3pt,
    minimum height=1.5em,
    minimum width=3em,
    inner sep=3pt,
    draw=index-border-color,
    line width=1.5pt,
    fill=index-color
  },
  git-workingtree/.style = {
    rectangle,
    rounded corners=3pt,
    minimum height=1.5em,
    minimum width=3em,
    inner sep=3pt,
    draw=wt-border-color,
    line width=1.5pt,
    fill=wt-color
  },
  git-head/.style = {
    rectangle,
    rounded corners=3pt,
    minimum height=1.5em,
    minimum width=3em,
    inner sep=3pt,
    draw=head-border-color,
    line width=1.5pt,
    fill=head-color
  },
  git-branch/.style = {
    rectangle,
    rounded corners=3pt,
    minimum height=1.5em,
    minimum width=3em,
    inner sep=3pt,
    draw=branch-border-color,
    line width=1.5pt,
    fill=branch-color
  }
}

%********** Misc **********
\usepackage{verbatim}
%%% for bookmarks
\usepackage{hyperref}
%********** hyperref settings **********
\hypersetup{
  bookmarksnumbered=true,
  bookmarksopen=true,
  pdfborder=1,
  breaklinks,
  colorlinks,
  linkcolor=cyan,
  filecolor=black,
  urlcolor=blue,
  citecolor=green
}


%********** title page settings **********
\title{Using \TeX{} for presentation}
%\subtitle{use beamer}
\author{ybin.sun}
\institute{\texttt{ybin.sun@gmail.com}}
\date{\today}

\begin{document}

%********** table of content **********
%\begin{frame}[plain]
%  \tableofcontents
%\end{frame}
%********** table of content **********


%\maketitle % 效果等同于 frame 0的\titlepage
\part[Title]{Title}
%*************** title page *****************
\begin{frame}[plain]
  \titlepage
\end{frame}
%*************** title page *****************


\part[Instruction]{Instruction}
\section[Greetings]{Greetings}
\begin{frame}{Greetings}
hello, beamer!
\end{frame}

\section[Source Code]{Source Code}
\begin{frame}{Source Code}
\lstinputlisting{example.c}
\end{frame}

\begin{frame}{Source Code 2}
  \verbatiminput{deletee.bat}
\end{frame}


\section[Chinese Fonts]{Chinese Fonts}
\begin{frame}{Chinese Fonts}
使用 \XeTeX{} 编译中文!
\end{frame}


\section[Control Flow]{Control Flow}
\begin{frame}{Control Flow}
\begin{figure}
  \centering
  \begin{tikzpicture}[node distance=3.5cm, auto, >=latex', thick]
%    \tikzstyle{box} = [
%      rectangle,
%      rounded corners=3pt,
%      minimum height=2.5em,
%      inner sep=3pt,
%      draw=gray, thick,
%      fill=lightgray
%    ]
    % 所谓bounding box即为告知TeX当前box的尺寸,TeX据此排版布局。
    % 所以,\centering告知TeX,当前box居中显示,而具体到实际的box
    % 里面的内容,则从坐标(0,0)开始布局。
    % 如果内容超出了bounding box的大小,没关系,TeX按照box的尺寸布局,
    % 图片超过部分按照box的坐标进行计算。
    \useasboundingbox (-1,-1) rectangle (8,1);
    % draw current bounding box
    %\draw[thick](current bounding box.south west) rectangle
    %            (current bounding box.north east);
    \path[->]<1-> node[box] (tex) {.tex};
    \path[->]<2-> node[box, right of=tex] (dvi) {.dvi}
                  (tex) edge node{latex} (dvi);
    \path[->]<3-> node[box, right of=dvi] (pdf) {.pdf}
                  (dvi) edge node{dvipdfmx} (pdf);
  \end{tikzpicture}
\end{figure}
\end{frame}

\begin{frame}{The \TeX\ work flow}
% 定义新的样式,名字为"format":
%   draw: 外框
%   fill: 填充色
%   ellipse: 椭圆形状
%\tikzstyle{format} = [draw, thin, fill=blue!20]
%\tikzstyle{medium} = [ellipse, draw, thin, fill=green!20, minimum height=2.5em]
\begin{figure}
  \begin{tikzpicture}[node distance=3cm, auto, >=latex', thick]
    % We need to set at bounding box first. Otherwise the diagram
    % will change position for each frame.
    \path[use as bounding box] (-1,0) rectangle (10,-2);
    % draw current bounding box
    %\draw[thick](current bounding box.south west) rectangle
    %            (current bounding box.north east);
    \path[->]<1-> node[format] (tex) {.tex file};
    \path[->]<2-> node[format, right of=tex] (dvi) {.dvi file}
                  (tex) edge node {\TeX} (dvi);
    \path[->]<3-> node[format, right of=dvi] (ps) {.ps file}
                  node[medium, below of=dvi] (screen) {screen}
                  (dvi) edge node {dvips} (ps)
                        edge node[swap] {xdvi} (screen);
    \path[->]<4-> node[format, right of=ps] (pdf) {.pdf file}
                  node[medium, below of=ps] (print) {printer}
                  (ps) edge node {ps2pdf} (pdf)
                       edge node[swap] {gs} (screen)
                       edge (print);
    \path[->]<5-> (pdf) edge (screen)
                        edge (print);
    \path[->, draw]<6-> (tex) -- +(0,1) -| node[near start] {pdf\TeX} (pdf);
  \end{tikzpicture}
\end{figure}
\end{frame}


\section[Picture Including]{Picture Including}
\begin{frame}{Insert Pictures}
\begin{figure}[htbp]
  \centering
  % Requires \usepackage{graphicx}
  \includegraphics[width=80pt]{sample.png}\\
  \caption{A sample picture \LaTeX}
  \label{fig:sample}
\end{figure}
\end{frame}


\part[TikZ Drawing]{TikZ Drawing}
\section[Basic]{Basic}
\begin{frame}{Basics}
\begin{figure}
  \centering
  \begin{tikzpicture}
    \draw (0,0) ellipse (2cm and 1cm)
                ellipse (0.5cm and 1cm)
                ellipse (0.5cm and 0.25cm);
  \end{tikzpicture}
\end{figure}
\end{frame}


\part[Local Operations]{Local Operations}
\begin{frame}
\end{frame}


\part[Remote Operations]{Remote Operations}
\begin{frame}
\end{frame}


\part[Tips \& Tricks]{Tips \& Tricks}
\begin{frame}
\end{frame}


\part[FAQ]{FAQ}
\begin{frame}
\end{frame}


\part[Reference]{Reference}

\end{document}
