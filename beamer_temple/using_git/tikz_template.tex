
\usepackage{xcolor}
\colorlet{HEADCOLOR}{red!50}
\colorlet{BRANCHCOLOR}{blue!50}
\colorlet{WORKCOLOR}{gray!50}
\colorlet{INDEXCOLOR}{cyan!50}
\colorlet{COMMITCOLOR}{green!50}

\usepackage{tikz}
\usetikzlibrary{shapes,arrows,positioning,calc,backgrounds,matrix,fit}
\tikzset{
  basic-style/.style = {
    rectangle, rounded corners=2pt, draw, thick,
    fill=#1,
    minimum height=15pt,
    minimum width=1.4cm,
    inner sep=1pt
  },
  commit-style/.style = {
    basic-style=COMMITCOLOR
  },
  index-style/.style = {
    basic-style=INDEXCOLOR,
    minimum width=2.5cm
  },
  work-style/.style = {
    basic-style=WORKCOLOR,
    minimum width=2.5cm
  },
  branch-style/.style = {
    basic-style=BRANCHCOLOR
  },
  head-style/.style = {
    basic-style=HEADCOLOR
  },
  cmd-style/.style = {
    #1=5pt
  },
  cmd-style/.default = right,
  main-style/.style = {
    execute at end picture = {
      \begin{pgfonlayer}{background}
        \path[fill=gray!20,rounded corners]
          ([xshift=-0.2cm,yshift=-0.2cm]current bounding box.south west) rectangle
          ([xshift=0.2cm,yshift=0.2cm]current bounding box.north east);
          %(current bounding box.south west) rectangle
          %  (current bounding box.north east);
        \end{pgfonlayer}
    }
  },
  file-style/.style = {
    draw=red, fill=yellow!30, inner sep=2pt
  },
  dir-style/.style = {
    fill=green!50,inner sep=2pt
  },
  dir-bg-style/.style = {
    fill=cyan!30,rounded corners
  },
  every edge/.style = {draw, ->, >=latex', thick}
}

%%%%% new definitions %%%%%

\newlength\commitDistance
\setlength\commitDistance{0.5cm}
\newlength\indexWorkDistance
\setlength\indexWorkDistance{2\commitDistance}

\newcommand\displayName[1]{\ttfamily\bfseries #1}
\newcommand\commandName[1]{\ttfamily\bfseries\small #1}

\def\createNode style:#1 name:#2 display:#3 direct:#4 distance:#5 to:#6;{
  \node [#1, #4=#5 of #6] (#2) {#3}
        edge [#4] (#6);
}
\def\createCommit name:#1 display:#2 direct:#3 to:#4;{
  \node [commit-style, #3=\commitDistance of #4] (#1) {#2}
        edge [#3, COMMITCOLOR] (#4);
}
\def\createBranch name:#1 display:#2 direct:#3 to:#4;{
  \node [branch-style, #3=0.75\commitDistance of #4] (#1) {#2}
        edge [#3, BRANCHCOLOR] (#4);
}
\def\createHead direct:#1 to:#2;{
  \node [head-style, #1=0cm of #2] (head) {HEAD};
}
\def\createIndex to:#1;{
  \node [index-style, below=2\commitDistance of #1] (index) {Index};
}
\def\createWork to:#1;{
  \node [work-style, below=2\commitDistance of #1] (work) {Work DIR};
}

\newcommand\makeOutline{
  %\useasboundingbox (-0.1,-3.6) rectangle (10.7,2);
  \node[right] (dumy) at (0,0) {\dots};
  \createCommit name:a display:{\displayName A} direct:right to:dumy;
  \createCommit name:b display:{\displayName B} direct:right to:a;
  \createCommit name:c display:{\displayName C} direct:right to:b;
  \createCommit name:d display:{\displayName D} direct:right to:c;
  \createCommit name:e display:{\displayName E} direct:right to:d;

  \createIndex to:c;
  \createWork to:index;
}

\newcommand\createMatrix[2]{
  \matrix [
    matrix of nodes,
    nodes={rectangle,draw=red,fill=yellow!30,minimum width=1.3cm,font=\ttfamily\small,inner sep=2pt},
    row sep=-\pgflinewidth,
    column sep=-\pgflinewidth,
  ] (m#1) at #2 {
    \textcolor{blue}{list-#1}\\
    a.h\\
    b.h\\
    c.h\\
    $d_{v#1}.h$\\
  };
}

\newcommand\createHashMatrix[2]{
  \matrix [
    matrix of nodes,
    nodes={rectangle,draw=red,fill=yellow!30,minimum width=1.3cm,font=\ttfamily\small,inner sep=2pt},
    row sep=-\pgflinewidth,
    column sep=-\pgflinewidth,
  ] (m#1) at #2 {
    \textcolor{blue}{hash-#1}\\
    a47c3\\
    b325c\\
    c10b9\\
    \textcolor{cyan}{da98#1}\\
  };
}
