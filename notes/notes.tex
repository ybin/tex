\documentclass{article}


%********** beamer settings **********
\usetheme{Madrid} %{default} %{Singapore} %{Warsaw} %{Copenhagen} %{Madrid}
\setbeamertemplate{navigation symbols}{} % remove the navigation symbols
%\useoutertheme{infolines} % use outer theme for footer.
\definecolor{mybackground}{RGB}{204,232,207} % note the uppercase RGB
%\setbeamercolor{normal text}{bg=red!12} % set background color
%\setbeamercolor{normal text}{bg=mybackground} % set background color
\definecolor{commit-color}{RGB}{204,255,204}
\definecolor{commit-border-color}{RGB}{160,254,160}
\definecolor{index-color}{RGB}{204,204,255}
\definecolor{index-border-color}{RGB}{178,178,254}
\definecolor{wt-color}{RGB}{255,204,204}
\definecolor{wt-border-color}{RGB}{255,153,153}
\definecolor{head-color}{RGB}{224,224,224}
\definecolor{head-border-color}{RGB}{192,192,192}
\definecolor{branch-color}{RGB}{255,224,192}
\definecolor{branch-border-color}{RGB}{255,192,128}

%********** fonts settings **********
\usepackage{fontspec,xunicode,xltxtra}
\setmainfont{Times New Roman}
\setsansfont{Arial} % beamer默认使用sans font
%\setmonofont{Source Sans Pro}

%********** new fonts commands **********
%\newfontfamily\SourceSansPro{Source Sans Pro}

%********** Chinese fonts input **********
\usepackage{xeCJK}
\setCJKmainfont{宋体}
\setCJKsansfont{微软雅黑}
\setCJKmonofont{宋体}
\XeTeXlinebreaklocale "zh"
\XeTeXlinebreakskip=0pt plus 1pt minus 0.1pt

%********** using packages **********
\usepackage{metalogo} % for XeTeX, XeLaTeX logo
\usepackage{listings} % for listing source code
% listings settings
\lstset{
  language=C,
  numbers=left,
  numberstyle=\scriptsize,
  frame=leftline,
  xleftmargin=1em,
  flexiblecolumns=false,
  basicstyle=\ttfamily\small,
  breaklines=true,
  extendedchars=true,
  escapechar=\%,
  texcl=true,
  showstringspaces=false,
  %keywordstyle=\bfseries,
  keywordstyle=\color{cyan},
  commentstyle=\tiny\color{blue}\slshape,
  tabsize=4
}

%********** colors and drawing pictures **********
%\usepackage{xcolor} % beamer 自动加载xcolor
\usepackage{graphicx}
\usepackage{tikz,ifthen}
\usetikzlibrary{arrows,shapes,fit,calc,positioning,intersections}
\tikzset{
  % 定义一些新的样式,注意每个样式定义后面的逗号
  box/.style={
    rectangle,
    rounded corners=3pt,
    minimum height=1.5em,
    minimum width=3em,
    inner sep=3pt,
    draw=gray,thick,
    fill=lightgray
  },
  format/.style = {draw, thin, fill=blue!20},
  medium/.style = {
    ellipse, draw, thin,
    fill=green!20,
    minimum height=2.5em
  },
  git-commit/.style = {
    rectangle,
    rounded corners=3pt,
    minimum height=1.5em,
    minimum width=3em,
    inner sep=3pt,
    draw=commit-border-color,
    line width=1.5pt,
    fill=commit-color
  },
  git-index/.style = {
    rectangle,
    rounded corners=3pt,
    minimum height=1.5em,
    minimum width=3em,
    inner sep=3pt,
    draw=index-border-color,
    line width=1.5pt,
    fill=index-color
  },
  git-workingtree/.style = {
    rectangle,
    rounded corners=3pt,
    minimum height=1.5em,
    minimum width=3em,
    inner sep=3pt,
    draw=wt-border-color,
    line width=1.5pt,
    fill=wt-color
  },
  git-head/.style = {
    rectangle,
    rounded corners=3pt,
    minimum height=1.5em,
    minimum width=3em,
    inner sep=3pt,
    draw=head-border-color,
    line width=1.5pt,
    fill=head-color
  },
  git-branch/.style = {
    rectangle,
    rounded corners=3pt,
    minimum height=1.5em,
    minimum width=3em,
    inner sep=3pt,
    draw=branch-border-color,
    line width=1.5pt,
    fill=branch-color
  }
}

%********** Misc **********
\usepackage{verbatim}
%%% for bookmarks
\usepackage{hyperref}
%********** hyperref settings **********
\hypersetup{
  bookmarksnumbered=true,
  bookmarksopen=true,
  pdfborder=1,
  breaklinks,
  colorlinks,
  linkcolor=cyan,
  filecolor=black,
  urlcolor=blue,
  citecolor=green
}


\sybtitle{\TeX{} family Notes}
\sybauthor{孙延宾}
\sybdate{\today}


\begin{document}
  \tt % I love Typewriter font.

%%%%%%%% the title page and toc %%%%%%%%%%
  \pagestyle{header}
  \sybmaketitle
  \tableofcontents
  \newpage

%%%%%%% the main content %%%%%%%%%
  \pagestyle{main}
  \setcounter{page}{1}

  \section[newif macro]{\bs newif macro}
  \bs newif 宏语法:
  \cmd{\bs newif\bs if\lt csname\gt}
  %\\[.5em]
  %\hspace*{4ex}\bs newif\bs if\lt csname\gt \\[.5em]
  该定义一次性定义了三个宏:
  \begin{description}
    \item[\bs if\lt csname\gt:] 条件判断语句
    \item[\bs\lt csname\gt true:] 调用这个宏会使得\bs if\lt csname\gt 中的条件判断为"true"
    \item[\bs\lt csname\gt false] 调用这个宏会使得\bs if\lt csname\gt 中的条件判断为"false"
  \end{description}
  下面是一个示例:

  \begin{latexcode}
\newif\ifboy
\newif\ifgirl
\ifboytrue
\ifboy{I love you, son!}\fi
\ifgirl{I love you, daughter!}\fi
  \end{latexcode}

  运行结果:
  \newif\ifboy
  \newif\ifgirl
  \boytrue
  \ifboy{I love you, son!}\fi
  \ifgirl{I love you, daughter!}\fi

  \section[loop macro]{loop macro}
  \bs loop宏的语法:
  \cmd{\bs loop\lt control sequences\gt \lt conditional sequence\gt \lt control sequences\gt \bs repeat}
  首先执行\bs loop后面的控制序列,然后,如果条件判断为true,就接着执行后面的控制序列,并循环之;
  如果条件判断为false,就退出循环。

  下面是一个嵌套循环的示例:

  \begin{latexcode}
\vbox{
  \count100=9
  \loop
    \count101=65 % ASCII `A'
    \advance\count100 by-1
    \hbox{% \loop的参数中不允许有\par等换行命令
          % 这里我们利用Tex的"模式"进行排版:
          % TeX默认进入vertical mode,然后使用\hbox进入stricted horizontal mode
          % 从而避免了换行命令。
      \loop
      \char\count101 \the\count100
      \advance\count101 by1
      \ifnum\count101<73
      \space
      \repeat
    }
  \ifnum\count100>0
  \repeat
}
  \end{latexcode}
  
  替换计数寄存器之后,代码更易读:
  
  \begin{latexcode}
\vbox{
  \newcount\num % \loop的参数中不能使用\newcount宏
  \newcount\chr % 不能使用\par宏
  \num=9
  \loop
    \chr=65 % ASCII `A'
    \advance\num by-1
    \hbox{% \loop的参数中不允许有\par等换行命令
          % 这里我们利用Tex的"模式"进行排版:
          % TeX默认进入vertical mode,然后使用\hbox进入stricted horizontal mode
          % 从而避免了换行命令。
      \loop
      \char\chr \the\num
      \advance\chr by1
      \ifnum\chr<73
      \space
      \repeat
    }
  \ifnum\num>0
  \repeat
}
  \end{latexcode}

  运行结果:\par\medskip
  \vbox{
    \count100=9
    \loop
      \count101=65 % ASCII `A'
      \advance\count100 by-1
      \hbox{% \loop的参数中不允许有\par等换行命令
            % 这里我们利用Tex的"模式"进行排版:
            % TeX默认进入vertical mode,然后使用\hbox进入stricted horizontal mode
            % 从而避免了换行命令。
        \loop
        \char\count101 \the\count100
        \advance\count101 by1
        \ifnum\count101<73
        \space
        \repeat
      }
    \ifnum\count100>0
    \repeat
  }

  %\expandafter\show\csname LaTeX \endcsname
\end{document}
